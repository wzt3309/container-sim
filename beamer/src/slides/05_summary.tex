%Title of section
\section{总结与展望}

\begin{frame}
\frametitle{总结与展望}
\begin{block}{总结}
\begin{itemize}
    \item 基于趋势感知的区间预测模型:SAC-GPSO-SVM
    \begin{itemize}
        \item {\color{gray}单值预测模型 $\Rightarrow$ \textbf{误差敏感,鲁棒性不足}}
        \item {\color{gray}传统区间预测模型 $\Rightarrow$ \textbf{先验假设,不考虑负载类型异构}}
        \item 负载趋势感知(SAC) $\Rightarrow$ \textbf{平稳型、趋势型和周期型}
        \item 针对类型构造训练区间 $\Rightarrow$ \textbf{SVM预测}(\sout{依赖分布假设})
        \item GPSO优化模型超参数 $\Rightarrow$ \textbf{实时性、准确性要求}
    \end{itemize}
    \item 基于动态伸缩的容器调度策略:DSCS
    \begin{itemize}
        \item 动态调整容器副本数量 $\Rightarrow$ \textbf{提升资源利用率,保证服务质量}
        \item 基于负载相关性容器和主机选择算法 $\Rightarrow$ \textbf{减少空闲主机,优化能耗}
    \end{itemize}
\end{itemize}
\end{block}
\begin{exampleblock}{展望}
    \begin{itemize}
        \item 容器接入雾计算和边缘计算:任务跨平台可移植性
        \item 容器调度系统复杂性和异构性:负载预测、资源配置和节能调度
    \end{itemize}
\end{exampleblock}
\end{frame}